\section{Tema}

\paragraph{}Falar do que se trata o trabalho usando uma visão macroscópica (tamanho do texto: 1 ou 2 parágrafos no máximo).

\paragraph{}Sobre que grande área de conhecimento você vai falar?

\paragraph{}Dada esta grande área, qual é o subconjunto de conhecimento sobre o qual será o seu trabalho?


\section{Delimitação}

\paragraph{}Realizar uma delimitação informando de quem é a demanda, em que local, e em que momento no tempo. Eventualmente, pode ser mais fácil começar pensando por exclusão, ou seja, para quem não serve, onde não deve ser aplicado, e em seguida pegar o universo que sobra (tamanho do texto: livre).


\section{Justificativa}

\paragraph{}Apresentar o porquê do tema ser interessante de ser estudado. Cuidado, não é a motivação particular. Devem ser apresentadas razões para que alguém deva se interessar no assunto, e não quais foram suas razões particulares que motivaram você a estudá-lo (tamanho do texto: livre).


\section{Objetivos}

\paragraph{}Informar qual é o objetivo geral do trabalho, isto é, aquilo que deve ser atendido e que corresponde ao indicador inequívoco do sucesso do seu trabalho. Pode acontecer que venha a existir um conjunto de objetivos específicos, que complementam o objetivo geral (tamanho do texto: livre, mas cuidado para não fazer uma literatura romanceada, afinal esta seção trata dos objetivos).


\section{Metodologia}

\paragraph{}Como é a abordagem do assunto. Como foi feita a pesquisa, se vai houve validação, etc. Em resumo, você de explicar qual foi sua estratégia para atender ao objetivo do trabalho (tamanho do texto: livre).


\section{Descrição}

\paragraph{}No capítulo 2 será .....

\paragraph{}O capítulo 3 apresenta ...

\paragraph{}Os .... são apresentados no capítulo 4. Nele será explicitado ...

\paragraph{}E assim vai até chegar na conclusão.
