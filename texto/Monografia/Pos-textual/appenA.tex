\chapter{Códigos Fonte dos Dispositivos Virtuais}
\label{apendiceA}

Seção do Docker Compose referente aos dispositivos virtuais.
\begin{lstlisting}[breaklines=true, basicstyle=\small\ttfamily]
services:
# ...
  virtual_device_1:
    container_name: device_1
    build:
      dockerfile: ${UBUNTU_DOCKERFILE}
      context: .
    image: ${UBUNTU_DEVICE_IMG}
    hostname: device1
    cpus: "0.6"
    mem_limit: "1024m"
    networks:
      - devices_network
    volumes:
      - /etc/timezone:/etc/timezone:ro
      - /etc/localtime:/etc/localtime:ro
    healthcheck:
      test: ["CMD", "curl", "-f", "http://localhost:9273/metrics"]
      interval: 30s
      timeout: 10s
      retries: 3

  virtual_device_2:
    container_name: device_2
    build:
      dockerfile: ${UBUNTU_DOCKERFILE}
      context: .
    image: ${UBUNTU_DEVICE_IMG}
    hostname: device2
    cpus: "0.6"
    mem_limit: "900m"
    networks:
      - devices_network
    volumes:
      - /etc/timezone:/etc/timezone:ro
      - /etc/localtime:/etc/localtime:ro
    healthcheck:
      test: ["CMD", "curl", "-f", "http://localhost:9273/metrics"]
      interval: 30s
      timeout: 10s
      retries: 3

  virtual_device_3:
    container_name: device_3
    build:
      dockerfile: ${UBUNTU_DOCKERFILE}
      context: .
    image: ${UBUNTU_DEVICE_IMG}
    hostname: device3
    cpus: "0.7"
    mem_limit: "800m"
    networks:
      - devices_network
    volumes:
      - /etc/timezone:/etc/timezone:ro
      - /etc/localtime:/etc/localtime:ro
    healthcheck:
      test: ["CMD", "curl", "-f", "http://localhost:9273/metrics"]
      interval: 30s
      timeout: 10s
      retries: 3
      
  virtual_device_4:
    container_name: device_4
    build:
      dockerfile: ${ALPINE_DOCKERFILE}
      context: .
    image: ${ALPINE_DEVICE_IMG}
    hostname: device4
    cpus: "0.5"
    mem_limit: "886m"
    networks:
      - devices_network
    volumes:
      - /etc/timezone:/etc/timezone:ro
      - /etc/localtime:/etc/localtime:ro
    healthcheck:
      test: ["CMD", "curl", "-f", "http://localhost:9273/metrics"]
      interval: 30s
      timeout: 10s
      retries: 3
      
  virtual_device_5:
    container_name: device_5
    build:
      dockerfile: ${ALPINE_DOCKERFILE}
      context: .
    image: ${ALPINE_DEVICE_IMG}
    hostname: device5
    cpus: "0.6"
    mem_limit: "750m"
    networks:
      - devices_network
    volumes:
      - /etc/timezone:/etc/timezone:ro
      - /etc/localtime:/etc/localtime:ro
    healthcheck:
      test: ["CMD", "curl", "-f", "http://localhost:9273/metrics"]
      interval: 30s
      timeout: 10s
      retries: 3
# ...
\end{lstlisting}

Dockerfile referente aos dispositivos virtuais Ubuntu (1 à 3):
\begin{lstlisting}[breaklines=true, basicstyle=\small\ttfamily]
FROM ubuntu:24.04

RUN apt update && apt install -y wget stress-ng gnupg curl iperf3

RUN wget -q https://repos.influxdata.com/influxdata-archive_compat.key

RUN echo '393e8779c89ac8d958f81f942f9ad7fb82a25e133faddaf92e15b16e6ac9ce4c influxdata-archive_compat.key' | sha256sum -c && cat influxdata-archive_compat.key | gpg --dearmor | tee /etc/apt/trusted.gpg.d/influxdata-archive_compat.gpg > /dev/null

RUN echo 'deb [signed-by=/etc/apt/trusted.gpg.d/influxdata-archive_compat.gpg] https://repos.influxdata.com/debian stable main' | tee /etc/apt/sources.list.d/influxdata.list

RUN apt update && apt install -y telegraf

COPY configs/telegraf/virtual-devices/telegraf.conf /etc/telegraf/telegraf.conf

COPY scripts/load_simulator.sh /usr/local/bin/load_simulator.sh

RUN chmod +x /usr/local/bin/load_simulator.sh

WORKDIR /

CMD ["/bin/bash", "-c", "/usr/local/bin/load_simulator.sh & telegraf --config /etc/telegraf/telegraf.conf"]
\end{lstlisting}

Dockerfile referente aos dispositivos virtuais Alpine (4 e 5):
\begin{lstlisting}[breaklines=true, basicstyle=\small\ttfamily]
FROM alpine:3.21.3

RUN apk update && apk add --no-cache wget stress-ng gnupg curl bash iperf3

RUN apk add --no-cache ca-certificates procps lm-sensors tzdata iputils 

RUN wget -q https://dl.influxdata.com/telegraf/releases/telegraf-1.29.4_linux_amd64.tar.gz && \
    tar -xzf telegraf-1.29.4_linux_amd64.tar.gz && \
    cp -r telegraf-1.29.4/* / && \
    rm -rf telegraf-1.29.4* && \
    mkdir -p /etc/telegraf

COPY configs/telegraf/virtual-devices/telegraf.conf /etc/telegraf/telegraf.conf

COPY scripts/load_simulator.sh /usr/local/bin/load_simulator.sh

RUN chmod +x /usr/local/bin/load_simulator.sh

WORKDIR /

CMD ["/bin/bash", "-c", "/usr/local/bin/load_simulator.sh & telegraf --config /etc/telegraf/telegraf.conf"]
\end{lstlisting}