\chapter{Conclusão}
\label{chap5}

{\color{red}
\section{Considerações Finais}
\label{section:ConsideracoesFinais}

Este trabalho atingiu com êxito o objetivo de desenvolver uma plataforma completa de monitoramento de dispositivos, capaz de coletar, armazenar, processar e apresentar visualizações e notificações de dados e métricas pertinentes à observabilidade de saturação de dispositivos. A solução proposta demonstrou eficácia no auxílio à detecção de gargalos sistêmicos e na identificação de oportunidades de otimização. Consequentemente, contribuiu para o aprimoramento do desempenho e da eficiência operacional dos dispositivos monitorados.

A arquitetura adotada, fundamentada em contêineres Docker, revelou-se flexível e escalável. Esta abordagem possibilitou uma implantação simplificada e facilitou a adaptação a diferentes ambientes operacionais. A implementação de paradigmas de Infraestrutura como Código acelerou os processos de configuração, manutenção, replicação, migração e recuperação da plataforma, resultando em redução significativa do tempo e dos recursos necessários para essas operações.

Adicionalmente, a adoção exclusiva de ferramentas de código aberto proporcionou duplo benefício: minimizou os custos associados ao desenvolvimento e operação da plataforma e assegurou elevado grau de personalização dos componentes utilizados. Esta estratégia resultou em maior adaptabilidade da solução às necessidades específicas do contexto de aplicação.

\section{Trabalhos Futuros}
\label{section:TrabalhosFuturos}

A integração de mecanismos de monitoramento para dispositivos móveis representa uma oportunidade significativa de desenvolvimento, especialmente considerando a ausência de suporte oficial para tais dispositivos.

Paralelamente, a implementação da solução em plataformas físicas dedicadas, como dispositivos Intel NUC ou Raspberry Pi, constituiria uma validação importante da viabilidade técnica e comercial da proposta. Tal implementação permitiria avaliar o desempenho em condições reais de operação e forneceria subsídios para o desenvolvimento de um eventual produto comercializável no mercado de IoT.

}