\chapter{Introdução}
\label{chap1}

\section{Tema}
\label{section:Tema}

Este projeto propõe o desenvolvimento de uma solução voltada ao monitoramento da saturação de recursos em dispositivos conectados a uma rede de computadores. Fundamentado exclusivamente em ferramentas de código aberto e na adoção do paradigma de Infraestrutura como Código (IaC)\abbrev{IaC}{\foreign{Infrastructure as Code}}, privilegia-se escalabilidade e replicabilidade nos processos de coleta, monitoramento e visualização de métricas. Assim, facilita-se a observabilidade da utilização do hardware dos dispositivos monitorados, com objetivo de auxiliar na administração e manutenção da infraestrutura da rede, seja ela corporativa ou doméstica.

\section{Delimitação}
\label{section:Delimitação}

A solução destaca-se em versatilidade, podendo ser empregada em diferentes cenários. Ela contempla desde o monitoramento de dispositivos em infraestruturas de tecnologia da informação (TI)\abbrev{TI}{Tecnologia da Informação} tradicionais --- como servidores, roteadores, \foreign{switches} e \foreign{storages} --- até o acompanhamento de equipamentos em ambientes domésticos, incluindo computadores pessoais, smartphones, dispositivos de Internet das Coisas (IoT)\abbrev{IoT}{\foreign{Internet of Things}} ou até mesmo dispositivos de computação de borda (\foreign{Edge computing}).

No entanto, diante da indisponibilidade de equipamentos físicos durante o desenvolvimento do projeto, adotou-se um escopo mais restrito. Para isso, implementou-se virtualmente uma rede doméstica composta por cinco \foreign{desktops}, utilizando contêi\-neres Docker com diferentes especificações de \foreign{Central Processing Unit} (CPU)\abbrev{CPU}{\foreign{Central Processing Unit}}, memória e sistema operacional. Paralelamente, optou-se por uma estratégia de isolamento dos recursos dos dispositivos virtuais, visando aproximar o comportamento desses ambientes simulados de um dispositivo físico real.

Além dos dispositivos simulados, incluiu-se também um computador físico, com o objetivo de enriquecer a análise e proporcionar uma base qualitativa para comparação das métricas obtidas nos dispositivos virtuais.

Como consequência das limitações impostas, inviabilizou-se a coleta de determinadas métricas dos dispositivos virtuais, especialmente aquelas relacionadas ao armazenamento, como espaço disponível e operações de entrada e saída (I/O)\abbrev{I/O}{\foreign{Input/Output}} em disco.

\section{Justificativa}
\label{section:Justificativa}

A relevância deste projeto reside na sua capacidade em atender uma crescente demanda por soluções de monitoramento eficazes, acessíveis, replicáveis e escaláveis de dispositivos conectados em redes heterogêneas. Ao empregar exclusivamente ferramentas de código aberto, a proposta democratiza o acesso a práticas avançadas de monitoramento, eliminando restrições impostas por soluções proprietárias e promovendo a adoção de padrões abertos e interoperáveis.

Sob a ótica da Engenharia de Confiabilidade de Sites (SRE)\abbrev{SRE}{\foreign{Site Reliability Engineering}}, destaca-se o conceito de saturação, que se refere à aproximação dos limites de capacidade de um recurso, como CPU, memória, armazenamento ou largura de banda.  A detecção proativa da saturação é fundamental para evitar falhas, degradações de desempenho e impactos negativos na experiência do usuário. O projeto oferece uma estrutura para a identificação dessas condições, por meio da coleta contínua e sistemática de métricas relevantes, possibilitando a implementação de ações preventivas e corretivas antes que o sistema atinja um estado crítico.

A adoção do paradigma de IaC constitui outro pilar central da proposta. Ao automatizar a definição, o provisionamento e o gerenciamento da infraestrutura de monitoramento por meio de código, o projeto assegura reprodutibilidade, versionamento e portabilidade, além de minimizar erros humanos e aumentar a eficiência operacional. Essa abordagem não só facilita a implantação da solução em múltiplos ambientes --- sejam eles físicos, virtuais ou em nuvem --- como também favorece a manutenção e a evolução contínua da infraestrutura monitorada, alinhando-se às melhores práticas contemporâneas de gestão de TI.

Por fim, o foco em observabilidade amplia a capacidade de compreensão do comportamento dos sistemas monitorados. Diferentemente do simples monitoramento, a observabilidade oferece uma visão holística e integrada dos dados coletados, permitindo a identificação proativa de anomalias, gargalos e tendências de saturação. Isso subsidia decisões informadas, baseadas em dados, e contribui para a otimização contínua do desempenho e da confiabilidade da infraestrutura.

Em síntese, ao integrar os princípios de saturação de SRE, IaC e observabilidade, este projeto não apenas supre uma lacuna técnica relevante, mas também promove a disseminação de práticas modernas e eficientes de gestão de infraestrutura de TI, em consonância com as demandas atuais por transparência, automação e sustentabilidade operacional.

\section{Objetivos}
\label{section:Objetivos}

O objetivo central deste projeto é o desenvolvimento de uma solução completa de monitoramento de saturação, capaz de coletar, armazenar e visualizar métricas de dispositivos numa rede, além de gerar alertas e notificações sempre que determinadas condições críticas forem detectadas. A solução deve ser facilmente adaptável a diferentes contextos, com arquitetura flexível, passível de ser reproduzida, ampliada e mantida de forma eficiente, além de utilizar apenas ferramentas de código aberto, garantindo a acessibilidade.

Para alcançar esse objetivo, é fundamental definir quais métricas são mais relevantes para o monitoramento de saturação, selecionar os softwares de coleta mais adequados, estabelecer o framework responsável pela captação e processamento dos dados, bem como determinar estratégias eficientes para o armazenamento e a visualização das informações. Além disso, é imprescindível a implementação de mecanismos de alerta e notificação que permitam a atuação rápida dos responsáveis cabíveis. Por fim, todo o processo deve ser estruturado de modo a assegurar portabilidade, escalabilidade, replicabilidade e versionamento, alinhando-se às melhores práticas de gestão de infraestrutura contemporânea.

\section{Metodologia}
\label{section:Metodologia}

No desenvolvimento deste trabalho, inicialmente foram avaliadas ferramentas de virtualização, coleta, armazenamento e visualização de dados e métricas, priorizando aquelas que atendessem aos objetivos propostos, apresentassem ampla documentação, ecossistema consolidado, recursos de automação, facilidade de uso e integração eficiente entre si.

Com isso em mãos, todos os serviços necessários à implementação foram virtualizados utilizando contêineres Docker, incluindo aqueles destinados à simulação dos dispositivos monitorados (dispositivos virtuais). A fim de simular máquinas distintas, cada dispositivo virtual recebeu restrições específicas de recursos de hardware, diferentes distribuições de Linux, uma instância dedicada e isolada do software coletor (agente), além de softwares executados periodicamente via \foreign{shell scripts} para realização de testes de estresse e carga, com o objetivo de simular cenários de saturação e gerar dados relevantes para análise.

Após a seleção dessas ferramentas, foram definidas as métricas mais relevantes para o monitoramento da saturação dos dispositivos, bem como os softwares de coleta mais adequados para captá-las, sempre considerando as limitações e restrições estabelecidas no escopo do projeto (ver seção \ref{section:Delimitação}).

Os agentes foram configurados para expor as métricas coletadas por meio de \foreign{endpoints} HTTP, possibilitando que o software de coleta centralizado capturasse, processasse e armazenasse os dados em um banco de dados. Esse repositório serviu de fonte para o \foreign{framework} de visualização, que disponibilizou \foreign{dashboards} interativos. Por fim, foram implementados mecanismos de alerta e notificação, responsáveis pelo envio automático de e-mails sempre que condições de disparo são atendidas.

\newpage

\section{Descrição}
\label{section:Descricao}

{\color{red}
O Capítulo \ref{chap2} apresenta os conceitos fundamentais necessários à compreensão das decisões de projeto, bem como as ferramentas utilizadas ou consideradas ao longo do desenvolvimento deste trabalho. O Capítulo \ref{chap3} detalha a solução proposta, descrevendo sua arquitetura e evolução de forma estruturada, além de abordar os desafios enfrentados e as estratégias adotadas para superá-los. No Capítulo \ref{chap4}, são analisadas em profundidade as visualizações obtidas e os mecanismos de notificação implementados. Por fim, o Capítulo \ref{chap5} expõe as conclusões do trabalho, apresentando também sugestões para trabalhos futuros e possíveis aprimoramentos da solução desenvolvida.
}